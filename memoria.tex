%% inicio, la clase del documento es iccmemoria.cls
\documentclass{iccmemoria}

%% datos generales y para la tapa
\titulo{T�tulo de la memoria}
\author{Autor}
\supervisor{Profesor gu�a}
\informantes
	{Profesor Informante 1}
	{Profesor Informante 2}
\adicional{(s�lo por si se necesita agregar alg�n otro profesor)}
\director{Profesor del ramo Memoria de T�tulo}
\date{mes, a�o}

\usepackage{verbatim}
\usepackage[printonlyused]{acronym}




%% inicio de documento
\begin{document}

%% crea la tapa
\maketitle

%% dedicatoria
\begin{dedicatory}
Dedicado a ...
\end{dedicatory}

%% agradecimientos
\begin{acknowledgment}
Agradecimientos a ...
\end{acknowledgment}

%% indices
\tableofcontents
\listoffigures
\listoftables

%% resumen
\begin{resumen}
Aqu� va el resumen (en Castellano)... 
\end{resumen}

%% abstract
\begin{abstract}
Here the abstract...
\end{abstract}

\chapter{Cap�tulo 1}

\section{Secci�n 1}



\section{Secci�n 2}


\chapter{Cap�tulo 2}
\section{Secci�n 1}



\section{Secci�n 2}


\chapter{Experimentos}

\chapter{Conclusiones y trabajo futuro}
\section{Conclusiones}
\section{Trabajo Futuro}


\section*{Glosario}
\newcommand{\ACRO}{\acro} %to include acronyms in the PDF. (and color
%\newcommand{\ACRO}{\acrodef}%to hide acronynms list. useful for last version.
\begin{acronym}[TDMA] %uncomment when using \acro
%Created By Cesar Astudillo (C) 2010
%if you find this file (and the technique) useful, 
% your are free to copy or modify, but do it at your own risk!
% this took me a lot of time and effort, so please leave this heading as credits :)
%-------------------------------------------------------------
%ACRONYMS for latex
%this file contains a series of acronyms used by the ''acronyms'' package.
%%%%%%%%%%%%%%%%%%%%%%%%%%%%%%%%%%%%%%%%%%%%%%%%%%%%%%%%%%%%%%%
%\begin{acronym}[TDMA]
\ACRO{ABSTSOM}{Adaptive \acs{BSTSOM}} 
\ACRO{ADS}{Adaptive Data Structure}
%\ACRO{AI}{Artificial Intelligence}
\ACRO{AI}{Inteligencia Artificial}
\ACRO{ART}{Adaptive Resonance Theory}
\ACRO{ANN}{Artificial Neural Network}
\ACRO{API}{Application Programming Interface}
\ACRO{AUC}{Area Under the ROC Curve}
\ACRO{BMU}{Best Matching Unit}
\ACRO{BN}{Bayesian Network} 
\ACRO{BoA}{Bubble of Activity}
\ACRO{BST}{Binary Search Tree}
\ACRO{BSTSOM}{Binary Search Tree \acs{SOM}} 
\ACRO{CL}{Competitive Learning}
\ACRO{CM}{Confusion Matrix}
\ACRO{CONROT-BST}{Conditional Rotations for a \acs{BST}}
\ACRO{CONROT}{Conditional Rotations}
\ACRO{CPT}{Conditional Probabilities Table}
\ACRO{CS}{Computer Science}
\ACRO{CV}{Computer Vision}
\ACRO{DAG}{Directed Acyclic Graph}
\ACRO{DS}{Data Structure}
\ACRO{DT}{D-Tree}
\ACRO{ET}{Evolving Tree}
\ACRO{GCS}{Growing Cell Structures}
\ACRO{GG}{Growing Grid}
\ACRO{GHSOM}{Growing Hierarchical SOM}
\ACRO{GHTSOM}{Growing Hierarchical Tree SOM}
\ACRO{GNG}{Growing Neural Gas}
\ACRO{GT}{Growth Threshold}
\ACRO{GSOM}{Growing SOM}
\ACRO{HFM}{Hierarchical Feature Map}
\ACRO{HST}{Hyperplane Search Tree}
\ACRO{HSTSOM}{Hyperplane Search Tree \acs{SOM}}
\ACRO{IB}{Instance-Based}
\ACRO{IDE}{Integrated Development Environment}
\ACRO{IID}{Independently and Identically Distributed}
\ACRO{IGG}{Incremental Grid Growing}
\ACRO{IR}{Information Retrieval}
\ACRO{JVM}{Java Virtual Machine}
\ACRO{JPD}{Joint Probability Distribution}
\ACRO{KNN}[$k$-NN]{$k$-Nearest Neighbor}
\ACRO{KCONROT}[$k$-CONROT]{\acs{CONROT} for $k$-ary trees}
\ACRO{LVQ}{Learning Vector Quantization}
\ACRO{MAP}{\textit{Maximum A Posteriori}}
\ACRO{ML}{Machine Learning}
\ACRO{MLE}{Maximum Likelihood Estimation}
\ACRO{MQE}{Mean Quantization Error}
\ACRO{MST}{Minimum Spanning Tree}
\ACRO{MT}{Monotonic Tree}
\ACRO{NB}{Na\"{i}ve Bayes}
\ACRO{NG}{Neural Gas}
\ACRO{NN}{Neural Network}
\ACRO{PCA}{Principal Component Analysis}
\ACRO{PDF}{Probability Density Function}
\ACRO{PM}{Pattern Matching}
\ACRO{PR}{Pattern Recognition}
\ACRO{QE}{Quantization Error}
\ACRO{RHST}{Random Hyperplane Search Tree}
\ACRO{ROC}{Receiver Operating Characteristics}
\ACRO{SOM}{Self-Organizing Maps}
\ACRO{SOTA}{Self-Organizing Tree Algorithm}
\ACRO{SOTM}{Self-Organizing Tree Map}
\ACRO{SVM}{Support Vector Machine}
\ACRO{TDD}{Test Driven Development}
\ACRO{TOD}{Threshold Order-Dependent}
\ACRO{TP}{Topographic Product}
\ACRO{TSVQ}{Tree-Structured VQ}
\ACRO{TSSOM}{Tree-Structured SOM}
\ACRO{TTOCONROT}{TTOSOM with Conditional Rotations}
\ACRO{TTOSOM}{Tree-based Topology Oriented SOM}
\ACRO{URL}{Uniform Resource Locator}
\ACRO{VQ}{Vector Quantization}
\ACRO{WDBC}{Wisconsin Diagnostic Breast Cancer}
\ACRO{WPL}{Weighted Path Length}
%\end{acronym}
%%%%%%%%%%%%%%%%%%%%%%%%%%%%%%%%%%%%%%%%%%%%%%%%%%%%%%%%%%%%%%%%

\end{acronym} %uncomment when using \acro\newpage


\bibliography{archivodereferencias}


\end{document}

   

